\chapter{Theoretical Foundations of Monotop Analysis}
\label{chapter:four}
This Chapter will focus on the explaining the theory behind the monotop analysis and the importance of the top quark.
\section{Monotop Analysis}

Great thoughts that further your argument. This includes lots of strong evidence presented throughout several paragraphs, each accompanied by necessary citations.
\begin{quotation}
    \noindent Here is a block quotation---a passage from a text you found insightful and wanted to share with others. Maybe it is from a journal article, website, or book. Irrespective, it should support the argument being made.\footnote{A citation for the quoted material.}
\end{quotation}
%Maybe a sentence or two that bring the argument and evidence together.\citep{dos_santos_2020}



\section{The Top quark}

The top quark was discovered the Fermilab Tevatron almost 25 years ago. It is the weak isospin partner of the bottom quark. Its discovery resulted in completion of the three generation structure of the standard model (SM). The top quark mass was measured to be $m_{t} = 176 \pm 13$ GeV, a charge of 2/3 times the charge on the electron and a lifetime of $5 \times 10^{-25} s$ making it the heaviest of all the fermions known till date in the standard model. Because of it's large mass and correspondingly short lifetime it behaves differently than other quarks. 

The top quark decays before it hadronizes, passing its information to the decay products. Therefore making it possible to infer its properties from the decay products in the detector. The top quark decays into a a bottom quark and a W boson, and since the W boson is an unstable particle it decays into a quark anti-quark pair of different flavors (hadronic decay channel) or to a charged lepton and a neutrino (leptonic decay channel) with the following branching ratios.

\begin{table}[h!]
  \centering
  \caption{}
  \label{tab:average_channels}
  \begin{tabular}{cc}
    \toprule
     Decay channel &  Branching ratio \\
     \midrule
      $W^{\pm} \rightarrow q\Bar{q'}$ &   $68.32\%$  \\
      $W^{\pm} \rightarrow e^{\pm} \Bar{\nu_{e}}$ &   $10.46\%$  \\
      $W^{\pm} \rightarrow \mu^{\pm} \Bar{\nu_{\mu}}$ &   $10.50\%$  \\
      $W^{\pm} \rightarrow \tau^{\pm} \Bar{\nu_{\tau}}$ &   $10.75\%$  \\
      \bottomrule
  \end{tabular}
\end{table}

As the top quark decays into a W boson and a bottom quark, it's evident that the hadronic and leptonic decay probabilities of the top quark will be proportional to the hadronic and leptonic decay probabilities of the W boson. So, it can be implied from the above table that, around 70\% of times the top quark decays into three quarks (two quarks from W boson and one bottom quark) which further hadronize to produce jets. While around 30\% of the times the top quark would decay leptonically to a lepton, its corresponding lepton neutrino and a bottom quark. In this thesis we will be exploring the leptonic channel for the search for Dark Matter (DM), so our final state signature would a lepton, its corresponding neutrino and a jet coming from a bottom quark.

In accordance with the SM, at the LHC, top quarks are predominantly produced in pairs ($t \Bar{t}$) through strong interactions and as single top via the electroweak interaction. In the following section we will be looking at the production of a single top quark (Monotop) in association with missing transverse energy due to the two DM candidates, as an extension to the SM. 


\section{The Monotop Model}