\chapter{Introduction}
\label{chapter:introduction}

%\section{An Interesting Section}
%\label{section:interesting_section}

The Standard Model (SM) of Particle Physics is one of the most complete and versatile theories which describes the fundamental particles and the forces with which they interact. The SM has been successful at describing many features of the nature that we have observed in our experiments. From the precise measurement of the electron's magnetic dipole moment to the discovery of the Higgs Boson. The predictions of the SM have just been too good. While the SM may has been really successful, there are some phenomenon like absence of the description of Dark Matter (DM) in the SM which prevents it from taking the title of the most complete and unified theory.

Dark Matter  is a form of material that neither emits, absorbs or reflects any Electromagnetic Radiation. It does seem to have mass given its Gravitational effects. Dark Matter contributes to around 24\% of the total mass energy of the universe (or 5 times the mass of ordinary matter) but does not interact directly with light, so we have not observed it yet. Astrophysical observations like Gravitational Lensing [2] (bending of light coming from distant galaxies by massive galaxies or galaxy clusters) and motion of galaxies at speeds so high for ordinary matter to sustain, Galaxy Cluster Collisions [3] and Dark Matter seeded Galaxy formations [4] have provided evidence for the presence of Dark Matter.

There are different ways in which Dark Matter could be detected-- direct detection, indirect detection and making Dark Matter on the Earth and then detecting it. The Large Hadron Collider (LHC) collides proton beams at the highest energy in the world. The proton beams collide at four main interaction points at the LHC. At each interaction point is a detector. In this thesis we will show the search for Dark Matter, based on the proton-proton collision data recorded by the Compact Muon Solenoid (CMS) experiment at the LHC.



\begin{quotation}
    \noindent Here is a block quotation---a passage from a text you found insightful and wanted to share with others. Maybe it is from a journal article, website, or book. Irrespective, it should support the argument being made.\footnote{A citation for the quoted material.}
\end{quotation}

%Maybe a sentence or two that bring the argument and evidence together.\citep{dos_santos_2020}

Check appendix \ref{appendix: chapter1}

\begin{figure} [ht]
\centering
         \includegraphics[width=0.9\textwidth,clip=]{Figures/Introduction/CUA_logo.png}
         \caption{CUA Logo}
         \label{CUA-logo-1}
\end{figure}


\section{Another Insightful Section}
\label{section:another_interesting_section}

More ideas that really make this a great paper. Maybe a footnote or two.\footnote{Some peripheral thoughts that belong in a note.}
